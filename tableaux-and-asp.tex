% ----------------------------------------------------------------------
\begin{frame}{Tableaux and ASP}
  \bigskip
  \begin{itemize}
  \item<1-> A tableau rule captures an elementary inference scheme in an\\ ASP solver
    \smallskip
  \item<2-> A \alert{branch} in a tableau corresponds to a successful or unsuccessful
    \alert{computation} of a stable model
    \smallskip
  \item<3-> An \alert{entire tableau} represents a traversal of the \alert{search space}
  \end{itemize}
\end{frame}
% ----------------------------------------------------------------------
\begin{frame}{ASP-specific definitions}
  \begin{itemize}
  \item<1-> A (signed) \alert{tableau} for a logic program $P$ is a binary tree such that
    \begin{itemize}\normalsize
    \item the root node of the tree consists of the rules in $P$
    \item the other nodes in the tree are \alert{entries} of the form \Tsigned{v}
      or \Fsigned{v}, called \alert{signed literals}, where $v$ is a variable,
    \item generated by extending a tableau using deduction rules (given below)
    \end{itemize}
    \smallskip
  \item<2-> An entry $\Tsigned{v}$ ($\Fsigned{v}$) reflects that variable $v$ is
    $\mathit{true}$ ($\mathit{false}$) in a corresponding variable assignment
  \item<2->[] A set of signed literals constitutes a partial assignment
    \smallskip
  \item<3-> For a normal logic program $P$,
    \begin{itemize}\normalsize
    \item atoms  of $P$ in $\atom{P}$ and
    \item bodies of $P$ in $\body{P}$
    \end{itemize}
    can occur as variables in signed literals
  \end{itemize}
\end{frame}
% ----------------------------------------------------------------------
\begin{frame}[shrink]{Tableau rules for ASP at a glance}
  \label{lab:tableau:rules:asp}
  \arrayrulewidth=.1pt
{\scriptsize % \ptsize{8} % \footnotesize
\[\hspace{-0.75cm}
\begin{array}{@{}rclp{5pt}rcl}
  \text{(FTB)}
  &
  \begin{array}{c}
    p \leftarrow l_1, \dots, l_n
    \\
    \plit{l_1},\dots,\plit{l_n}
    \\\hline
    \Tsigned{\{l_1, \dots, l_n\}}
  \end{array}
  &
  &&
  \text{(BFB)}
  &
  \begin{array}{c}
    \Fsigned{\{l_1, \dots, l_i, \dots, l_n\}}
    \\
    \plit{l_1},\dots,\plit{l_{i-1}}
    ,
    \plit{l_{i+1}},\dots,\plit{l_n}
    \\\hline
    \nlit{l_i}
  \end{array}
  &
  \\[12pt]
  \text{(FTA)}
  &
  \begin{array}{c}
    p \leftarrow l_1, \dots, l_n
    \\
    \Tsigned{\{l_1, \dots, l_n\}}
    \\\hline
    \Tsigned{p}
  \end{array}
  &
  &&
  \text{(BFA)}
  &
  \begin{array}{c}
    p \leftarrow l_1, \dots, l_n
    \\
    \Fsigned{p}
    \\\hline
    \Fsigned{\{l_1, \dots, l_n\}}
  \end{array}
  &
  \\[12pt]
  \text{(FFB)}
  &
  \begin{array}{c}
    p \leftarrow l_1, \dots, l_i, \dots, l_n
    \\
    \nlit{l_i}
    \\\hline
    \Fsigned{\{l_1, \dots, l_i, \dots, l_n\}}
  \end{array}
  &
  &&
  \text{(BTB)}
  &
  \begin{array}{c}
    \Tsigned{\{l_1, \dots, l_i, \dots, l_n\}}
    \\\hline
    \plit{l_i}
  \end{array}
  &
  \\[12pt]
  \text{(FFA)}
  &
  \begin{array}{c}
    \Fsigned{B_1},\dots,\Fsigned{B_m}
    \\\hline
    \Fsigned{p}
  \end{array}
  & \Proviso{\S}
  &&
  \text{(BTA)}
  &
  \begin{array}{c}
    \Tsigned{p}
    \\
    \Fsigned{B_1},\dots,\Fsigned{B_{i-1}}
    ,
    \Fsigned{B_{i+1}},\dots,\Fsigned{B_m}
    \\\hline
    \Tsigned{B_i}
  \end{array}
  & \Proviso{\S}
  \\[12pt]
  \text{(WFN)}
  &
  \begin{array}{c}
    \Fsigned{B_1},\dots,\Fsigned{B_m}
    \\\hline
    \Fsigned{p}
  \end{array}
  & \Proviso{\dagger}
  &&
  \text{(WFJ)}
  &
  \begin{array}{c}
    \Tsigned{p}
    \\
    \Fsigned{B_1},\dots,\Fsigned{B_{i-1}}
    ,
    \Fsigned{B_{i+1}},\dots,\Fsigned{B_m}
    \\\hline
    \Tsigned{B_i}
  \end{array}
  & \Proviso{\dagger}
  \\[12pt]
  \text{(FL)}
  &
  \begin{array}{c}
    \Fsigned{B_1},\dots,\Fsigned{B_m}
    \\\hline
    \Fsigned{p}
  \end{array}
  & \Proviso{\ddagger}
  &&
  \text{(BL)}
  &
  \begin{array}{c}
    \Tsigned{p}
    \\
    \Fsigned{B_1},\dots,\Fsigned{B_{i-1}}
    ,
    \Fsigned{B_{i+1}},\dots,\Fsigned{B_m}
    \\\hline
    \Tsigned{B_i}
  \end{array}
  & \Proviso{\ddagger}
  \\[5pt]
  \multicolumn{7}{@{}c}{%
  \begin{array}{@{}rcl}
    \text{(Cut[$X$])}
    &
    \begin{array}{c}
     \mbox{}
     \\\hline
     \Tsigned{v} \;\;\; | \;\;\; \Fsigned{v}
    \end{array}
    & \Proviso{\sharp[X]}
  \end{array}}
%   \\[12pt]
%   \multicolumn{7}{c}{%
%   \begin{array}[t]{rlp{1pt}rl}
%     (\S)& \body{p} = \{B_1,\dots,B_m\}
%     &&
%     (\dagger)& \{B_1,\dots,B_m\}\subseteq\body{P},
%                  \ p \in \gus{\{r\in P\mid
%                               \body{r}\not\in\{B_1,\dots,B_m\}\},\emptyset}
%     \\
%     (\ddagger)& p \in L, L \in \loops{P},
%                   \ \extbody{L} = \{B_1,\dots,B_m\}
%     &&
%     (\sharp[X])& v \in X
%   \end{array}}
\end{array}
\]}
% \scriptsize
% See paper for formal details on provisos $(\S)$, $(\dagger)$, $(\ddagger)$, and $(\sharp[X])$.
%%% Local Variables:
%%% mode: latex
%%% TeX-master: "../asp"
%%% End:

\end{frame}
% ----------------------------------------------------------------------
\begin{frame}{More concepts}
  \begin{itemize}
  \item <1-> A \alert{tableau calculus} is a set of tableau rules
  \item <2-> A branch in a tableau is \alert{conflicting},\\
    if it contains both $\Tsigned{v}$ and $\Fsigned{v}$ for some variable $v$
  \item <3-> A branch in a tableau is \alert{total} for a program $P$,\\
    if it contains either $\Tsigned{v}$ or $\Fsigned{v}$ for each $v \in \atom{P} \cup \body{P}$
  \item <4-> A branch in a tableau of some calculus $\mathcal{T}$ is \alert{closed},\\
    if no rule in $\mathcal{T}$ other than \textit{Cut} can produce any new entries
  \item <5-> A branch in a tableau is \alert{complete},\\
    if it is either conflicting or both total and closed
  \item <6-> A tableau is \alert{complete}, if all its branches are complete
  \item <7-> A tableau of some calculus $\mathcal{T}$ is a \alert{refutation} of $\mathcal{T}$ for a program $P$,\\
    if every branch in the tableau is conflicting
  \end{itemize}
\end{frame}
% ----------------------------------------------------------------------
\begin{frame}[shrink]{Previewed example}
\[
\begin{array}{@{}rc}
&a\leftarrow              \\
&c\leftarrow \neg b,\neg d  \\
&d\leftarrow     a,\neg c
\\[3pt]
\onslide<2->(\mathsf{FTB})&\onslide<2->\Tsigned{\emptyset}         \\
\onslide<2->(\mathsf{FTA})&\onslide<2->\alert<6>{\Tsigned{a}} \\
\onslide<2->(\mathsf{FFA})&\onslide<2->\Fsigned{b}
\\
\onslide<3->(\mathsf{Cut}[\atom{P}])&
\begin{array}[t]{@{}rcp{30pt}rc}
                 &\onslide<3->\alert<6>{\Tsigned{c}}         & &               & \onslide<3->\Fsigned{c}                 \pause\\
 \onslide<4->(\mathsf{BTA})  &\onslide<4->\Tsigned{\{\neg b,\neg d\}}   & & \onslide<5->(\mathsf{BFA})&
                                                                                                        \onslide<5->\Fsigned{\{\neg
                                                                                                        b,\neg d\}}         \\
 \onslide<4->(\mathsf{BTB})  &\onslide<4->\Fsigned{d}                 & & \onslide<5->(\mathsf{BFB})& \onslide<5->\alert<6>{\Tsigned{d}}       \\
 \onslide<4->(\mathsf{FFB})  &\onslide<4->\Fsigned{\{a,\neg c\}}       & & \onslide<5->(\mathsf{FTB})&
                                                                                                       \onslide<5->\Tsigned{\{a,\neg c\}}
\end{array}
\end{array}
\]
\bigskip
The branches correspond to stable models $\{a,c\}$ and $\{a,d\}$
\end{frame}
% ------------------------------------------------------------
\begin{frame}{Auxiliary definitions}
  \bigskip
  \begin{itemize}
  \item<1->
    For a literal $l$, define conjugation functions \plit{} and \nlit{} as follows
    \[
    \begin{array}{rcl}
      \plit{l}
      &=&
      \left\{
        \begin{array}{ll}
          \Tsigned{l}&\text{if $l$ is an atom} % l\in\mathcal{A}
          \\
          \Fsigned{a}&\text{if $l=\neg a$ for an atom $a$} % a\in\mathcal{A}
        \end{array}
      \right.
      \\[20pt]
      \nlit{l}
      &=&
      \left\{
        \begin{array}{ll}
          \Fsigned{l}&\text{if $l$ is an atom} %  l\in\mathcal{A}
          \\
          \Tsigned{a}&\text{if $l=\neg a$ for an atom $a$} % a\in\mathcal{A}
        \end{array}
      \right.
    \end{array}
    \]
  \item<2-> \structure{Examples}
    $\plit{a}=\Tsigned{a}$, $\nlit{a}=\Fsigned{a}$, $\plit{\neg a}=\Fsigned{a}$, and $\nlit{\neg a}=\Tsigned{a}$
\end{itemize}
\end{frame}
% ------------------------------------------------------------
\begin{frame}{Auxiliary definitions}
  \begin{itemize}
  \item Some tableau rules require conditions for their application
  \item Such conditions are specified as \alert{provisos}
    \[
    \begin{array}{lcr}
      \begin{array}{cl}
        \textit{prerequisites} &
        \raisebox{-8pt}[0pt][0pt]{(\textit{proviso})}
        \\\cline{1-1}
        \textit{consequence}
      \end{array}
      & \hspace{1.5cm} &
      \textit{proviso\/}{:}\text{ some condition(s)}
    \end{array}
    \]
    \bigskip
  \item \structure{Note} All tableau rules given in the sequel are stable model preserving
  \end{itemize}
\end{frame}
%----------------------------------------------------------------
\begin{frame}{Forward true body (FTB)}
\begin{itemize}
\item \structure{Prerequisites} All of a body's literals are $\mathit{true}$
\item \structure{Consequence} The body is $\mathit{true}$
\item \structure{Tableau Rule FTB}
\[
\begin{array}{c}
p \leftarrow l_1,\dots,l_n \\
\plit{l_1},\dots,\plit{l_n} \\\hline
\Tsigned{\{l_1,\dots,l_n\}}
\end{array}
\]
\item<2-> \structure{Example}
\[
\begin{array}{c}
a \leftarrow b, \neg c \\
\Tsigned{b} \\
\Fsigned{c} \\\hline
\Tsigned{\{b, \neg c\}}
\end{array}
\]
\end{itemize}
\end{frame}
%----------------------------------------------------------------
\begin{frame}{Backward false body (BFB)}
\begin{itemize}
\item \structure{Prerequisites}
  A body is $\mathit{false}$, and all its literals except for one are $\mathit{true}$
\item \structure{Consequence} The residual body literal is $\mathit{false}$
\item \structure{Tableau Rule BFB}
\[
\begin{array}{c}
\Fsigned{\{l_1,\dots,l_i,\dots,l_n\}} \\
\plit{l_1},\dots,\plit{l_{i-1}},\plit{l_{i+1}},\dots,\plit{l_n} \\\hline
\nlit{l_i}
\end{array}
\]
\item<2-> \structure{Examples}
\[
\begin{array}{c}
\Fsigned{\{b, \neg c\}} \\
\Tsigned{b} \\\hline
\Tsigned{c}
\end{array}
\qquad\qquad
\begin{array}{c}
\Fsigned{\{b, \neg c\}} \\
\Fsigned{c} \\\hline
\Fsigned{b}
\end{array}
\]
\end{itemize}
\end{frame}
%----------------------------------------------------------------
\begin{frame}{Forward false body (FFB)}
\begin{itemize}
\item \structure{Prerequisites} Some literal of a body is $\mathit{false}$
\item \structure{Consequence} The body is $\mathit{false}$
\item \structure{Tableau Rule FFB}
\[
\begin{array}{c}
p \leftarrow l_1,\dots,l_i,\dots,l_n \\
\nlit{l_i} \\\hline
\Fsigned{\{l_1,\dots,l_i,\dots,l_n\}}
\end{array}
\]
\item<2-> \structure{Examples}
\[
\begin{array}{c}
a \leftarrow b, \neg c \\
\Fsigned{b} \\\hline
\Fsigned{\{b, \neg c\}}
\end{array}
\qquad\qquad
\begin{array}{c}
a \leftarrow b, \neg c \\
\Tsigned{c} \\\hline
\Fsigned{\{b, \neg c\}}
\end{array}
\hfill
\]
\end{itemize}
\end{frame}
%----------------------------------------------------------------
\begin{frame}{Backward true body (BTB)}
\begin{itemize}
\item \structure{Prerequisites} A body is $\mathit{true}$
\item \structure{Consequence} The body's literals are $\mathit{true}$
\item \structure{Tableau Rule BTB}
\[
\begin{array}{c}
\Tsigned{\{l_1,\dots,l_i,\dots,l_n\}} \\\hline
\plit{l_i}
\end{array}
\]
\item<2-> \structure{Examples}
\[
\begin{array}{c}
\Tsigned{\{b, \neg c\}} \\\hline
\Tsigned{b}
\end{array}
\qquad\qquad
\begin{array}{c}
\Tsigned{\{b, \neg c\}} \\\hline
\Fsigned{c}
\end{array}
\hfill
\]
\end{itemize}
\end{frame}
%----------------------------------------------------------------
\begin{frame}{Tableau rules for bodies}
Consider rule body $B=\{l_1,\dots,l_n\}$
\bigskip
\begin{itemize}
\item<1-> Rules FTB and BFB amount to implication
  \[
  l_1 \wedge \dots \wedge l_n \rightarrow B
  \]
\item<2-> Rules FFB and BTB amount to implication
  \[
  B \rightarrow l_1 \wedge \dots \wedge l_n
  \]
\item<3-> Together they yield
  \[
  B \leftrightarrow l_1 \wedge \dots \wedge l_n
  \]
\end{itemize}
\end{frame}
%----------------------------------------------------------------
\begin{frame}{Forward true atom (FTA)}
\begin{itemize}
\item \structure{Prerequisites} Some of an atom's bodies is $\mathit{true}$
\item \structure{Consequence} The atom is $\mathit{true}$
\item \structure{Tableau Rule FTA}
\[
\begin{array}{c}
p \leftarrow l_1,\dots,l_n \\
\Tsigned{\{l_1,\dots,l_n\}} \\\hline
\Tsigned{p}
\end{array}
\]
\item<2-> \structure{Examples}
\[
\begin{array}{c}
a \leftarrow b, \neg c \\
\Tsigned{\{b, \neg c\}} \\\hline
\Tsigned{a}
\end{array}
\qquad\qquad
\begin{array}{c}
a \leftarrow d, \neg e \\
\Tsigned{\{d, \neg e\}} \\\hline
\Tsigned{a}
\end{array}
\hfill
\]
\end{itemize}
\end{frame}
%----------------------------------------------------------------
\begin{frame}{Backward false atom (BFA)}
\begin{itemize}
\item \structure{Prerequisites} An atom is $\mathit{false}$
\item \structure{Consequence} The bodies of all rules with the atom as head are $\mathit{false}$
\item \structure{Tableau Rule BFA}
\[
\begin{array}{c}
p \leftarrow l_1,\dots,l_n \\
\Fsigned{p} \\\hline
\Fsigned{\{l_1,\dots,l_n\}}
\end{array}
\]
\item<2-> \structure{Examples}
\[
\begin{array}{c}
a \leftarrow b, \neg c \\
\Fsigned{a} \\\hline
\Fsigned{\{b, \neg c\}}
\end{array}
\qquad\qquad
\begin{array}{c}
a \leftarrow d, \neg e \\
\Fsigned{a} \\\hline
\Fsigned{\{d, \neg e\}}
\end{array}
\hfill
\]
\end{itemize}
\end{frame}
%----------------------------------------------------------------
\begin{frame}{Forward false atom (FFA)}
\begin{itemize}
\item \structure{Prerequisites} For some atom, the bodies of all rules with the atom as head are $\mathit{false}$
\item \structure{Consequence} The atom is $\mathit{false}$
\item \structure{Tableau Rule FFA}
\[
\begin{array}{cl}
\Fsigned{B_1},\dots,\Fsigned{B_m} &
\raisebox{-8pt}[0pt][0pt]{($\atbody{P}{p} = \{B_1,\dots,B_m\}$)}
\\\cline{1-1}
\Fsigned{p}
\end{array}
\]
% \item\structure{Note} \ $\atbody{P}{a}=\{\body{r}\mid r\in P,\head{r}=a\}$
\item<2-> \structure{Example}
\[
\begin{array}{cl}
\Fsigned{\{b,\neg c\}} \\
\Fsigned{\{d,\neg e\}} &
\raisebox{-8pt}[0pt][0pt]{($\atbody{P}{a} = \{\{b,\neg c\},\{d,\neg e\}\}$)}
\\\cline{1-1}
\Fsigned{a}
\end{array}
\]
\end{itemize}
\end{frame}
%----------------------------------------------------------------
\begin{frame}{Backward true atom (BTA)}
\begin{itemize}
\item \structure{Prerequisites}
  An atom is $\mathit{true}$, and the bodies of all rules with the atom as head except for one are $\mathit{false}$
\item \structure{Consequence} The residual body is $\mathit{true}$
\item \structure{Tableau Rule BTA}
\[
\begin{array}{cl}
\Tsigned{p} \\
\Fsigned{B_1},\dots,\Fsigned{B_{i-1}},\Fsigned{B_{i+1}},\dots,\Fsigned{B_m} &
\raisebox{-8pt}[0pt][0pt]{($\atbody{P}{p} = \{B_1,\dots,B_m\}$)}
\\\cline{1-1}
\Tsigned{B_i}
\end{array}
\]
\item<2-> \structure{Examples}
\[
\begin{array}{cl}
\Tsigned{a} \\
\Fsigned{\{b,\neg c\}} &
\raisebox{-8pt}[0pt][0pt]{$(*)$}
\\\cline{1-1}
\Tsigned{\{d,\neg e\}}
\end{array}
\qquad\qquad
\begin{array}{cl}
\Tsigned{a} \\
\Fsigned{\{d,\neg e\}} &
\raisebox{-8pt}[0pt][0pt]{$(*)$}
\\\cline{1-1}
\Tsigned{\{b,\neg c\}}
\end{array}
\hfill
\]
\[
\hfill
(*)\quad \atbody{P}{a} = \{\{b,\neg c\},\{d,\neg e\}\}
\hfill\hfill
\]
%\vspace{-1cm}
\end{itemize}
\end{frame}
%----------------------------------------------------------------
\begin{frame}{Tableau rules for atoms}
Consider an atom~$p$ such that $\atbody{P}{p}=\{B_1,\dots,B_m\}$
\bigskip
\begin{itemize}
\item<1-> Rules FTA and BFA amount to implication
  \[
  B_1 \vee \dots \vee B_m \rightarrow p
  \]
\item<2-> Rules FFA and BTA amount to implication
  \[
  p \rightarrow B_1 \vee \dots \vee B_m
  \]
\item<3-> Together they yield
  \[
  p \leftrightarrow B_1 \vee \dots \vee B_m
  \]
\end{itemize}
\end{frame}
%----------------------------------------------------------------
\begin{frame}{Relationship with program completion}
  \bigskip
  Let $P$ be a normal logic program

  \begin{itemize}
  \item <1-> The eight tableau rules introduced so far essentially provide
    (straightforward) inferences from $\CF{P}$  % \end{itemize}
  \end{itemize}
\end{frame}
% ----------------------------------------------------------------
\begin{frame}{Preliminaries for unfounded sets}
  \bigskip
  Let $P$ be a normal logic program
  \begin{itemize}
\item<1-> For $P'\subseteq P$, define the \alert{greatest unfounded set} of $P$ wrt $P'$ as
\[
\mathbf{U}_P(P') = \atom{P}\setminus \Cn{\reduct{(P')}{\emptyset}}
\]
% \begin{itemize}
% \item \structure{Note} The rule-based definition of the greatest unfounded set
%   is more flexible than the atom-based one
% \end{itemize}
\item<2-> For a loop $L \in \Loops{P}$, define the \alert{external bodies} of $L$ as
\[
\EB{L}{P} = \{\body{r} \mid r \in P, \head{r} \in L, \pbody{r} \cap L = \emptyset\}
\]
\end{itemize}
\end{frame}
%----------------------------------------------------------------
\begin{frame}{Well-founded negation (WFN)}
\begin{itemize}
\item \structure{Prerequisites} An atom is in the greatest unfounded set wrt rules whose bodies are $\mathit{false}$
\item \structure{Consequence} The atom is $\mathit{false}$
\item \structure{Tableau Rule WFN}
\[
\hspace{-5mm}
\begin{array}{@{}cl}
\Fsigned{B_1},\dots,\Fsigned{B_m} &
\raisebox{-8pt}[0pt][0pt]{($p \in \mathbf{U}_P(\{r \in P \mid \body{r} \not\in \{B_1,\dots,B_m\}\})$)}
\\\cline{1-1}
\Fsigned{p}
\end{array}
\]
\item<2-> \structure{Examples}
\[
\begin{array}[b]{cl}
a \leftarrow \neg b \\
\Fsigned{\{\neg b\}} &
\raisebox{-8pt}[0pt][0pt]{$(*)$}
\\\cline{1-1}
\Fsigned{a}
\end{array}
\qquad\qquad
\begin{array}[b]{cl}
a \leftarrow a \\
a \leftarrow \neg b \\
\Fsigned{\{\neg b\}} &
\raisebox{-8pt}[0pt][0pt]{$(*)$}
\\\cline{1-1}
\Fsigned{a}
\end{array}
\hfill
\]
\[
\hfill
(*)\quad a \in \mathbf{U}_P(P \setminus \{a \leftarrow \neg b\})
\hfill\hfill
\]
\end{itemize}
\end{frame}
%----------------------------------------------------------------
\begin{frame}{Well-founded justification (WFJ)}
\begin{itemize}
\item \structure{Prerequisites} A $\mathit{true}$ atom is in the greatest unfounded set wrt rules whose bodies are $\mathit{false}$,
  if a particular body is made $\mathit{false}$
\item \structure{Consequence} The respective body is $\mathit{true}$
\item \structure{Tableau Rule WFJ} %~\\[-4mm]
\[
\hspace{-10mm}
{\small
\begin{array}{@{}cl}
\Tsigned{p} \\
\Fsigned{B_1},\dots,\Fsigned{B_{i-1}},\Fsigned{B_{i+1}},\dots,\Fsigned{B_m} &
\raisebox{-8pt}[0pt][0pt]{($p \in \mathbf{U}_P(\{r \in P \mid \body{r} \not\in \{B_1,\dots,B_m\}\})$)}
\\\cline{1-1}
\Tsigned{B_i}
\end{array}
}
\]
\item<2-> \structure{Examples} %~\\[-5mm]
\[
\begin{array}[b]{cl}
a \leftarrow \neg b \\
\Tsigned{a} &
\raisebox{-8pt}[0pt][0pt]{$(*)$}
\\\cline{1-1}
\Tsigned{\{\neg b\}}
\end{array}
\qquad\qquad
\begin{array}[b]{cl}
a \leftarrow a \\
a \leftarrow \neg b \\
\Tsigned{a} &
\raisebox{-8pt}[0pt][0pt]{$(*)$}
\\\cline{1-1}
\Tsigned{\{\neg b\}}
\end{array}
\hfill
\]
\[
\hfill
(*)\quad a \in \mathbf{U}_P(P \setminus \{a \leftarrow \neg b\})
\hfill\hfill
\]
%\vspace{-1cm}
\end{itemize}
\end{frame}
%----------------------------------------------------------------
\begin{frame}{Well-founded tableau rules}
  \bigskip
  \begin{itemize}
  \item<1-> Tableau rules WFN and WFJ ensure non-circular support for $\mathit{true}$ atoms
  \item<2-> \structure{Note}
    \begin{itemize}\normalsize
    \item WFN subsumes falsifying atoms via FFA,
    \item WFJ can be viewed as ``backward propagation'' for unfounded sets,
    \item WFJ subsumes backward propagation of $\mathit{true}$ atoms via BTA
    \end{itemize}
  \end{itemize}
\end{frame}
%----------------------------------------------------------------
\begin{frame}{Relationship with well-founded operator}
\bigskip
Let $P$ be a normal logic program,
$\langle T,F \rangle$ a partial interpretation, and\\
$P'=\{r\in P \mid \pbody{r}\cap F=\emptyset \text{ and } \nbody{r}\cap T=\emptyset\}$.\\
\bigskip
\begin{itemize}
\item <2-> The following conditions are equivalent
  \begin{itemize}\normalsize
  \item $p\in\mathbf{U}_P\langle T,F \rangle$
  \item $p\in\mathbf{U}_P(P')$
  \end{itemize}
  \medskip
\item<3-> Hence, the well-founded operator $\mathbf{\Omega}$ and WFN coincide
\item<4-> \structure{Note} In contrast to $\mathbf{\Omega}$, WFN does not necessarily require a rule
  body to contain a $\mathit{false}$ literal for the rule being inapplicable
\end{itemize}
\end{frame}
%----------------------------------------------------------------
\begin{frame}{Forward loop (FL)}
\begin{itemize}
\item \structure{Prerequisites} The external bodies of a loop are $\mathit{false}$
\item \structure{Consequence} The atoms in the loop are $\mathit{false}$
\item \structure{Tableau Rule FL}
\[
\begin{array}{@{}cl}
\Fsigned{B_1},\dots,\Fsigned{B_m} &
\raisebox{-8pt}[0pt][0pt]{($p \in L, L \in \Loops{P}, \EB{L}{P}=\{B_1,\dots,B_m\}$)}
\\\cline{1-1}
\Fsigned{p}
\end{array}
\]
\item<2-> \structure{Example}
\[
\begin{array}{cl}
a \leftarrow a \\
a \leftarrow \neg b \\
\Fsigned{\{\neg b\}} &
\raisebox{-8pt}[0pt][0pt]{($\EB{\{a\}}{P}=\{\{\neg b\}\}$)}
\\\cline{1-1}
\Fsigned{a}
\end{array}
\]
%\vspace{-1cm}
\end{itemize}
\end{frame}
%----------------------------------------------------------------
\begin{frame}{Backward loop (BL)}
\begin{itemize}
\item \structure{Prerequisites} An atom of a loop is $\mathit{true}$,
  and all external bodies except for one are $\mathit{false}$
\item \structure{Consequence} The residual external body is $\mathit{true}$
\item \structure{Tableau Rule BL}
\[
\hspace{-10mm}
{\small
\begin{array}{@{}cl}
\Tsigned{p} \\
\Fsigned{B_1},\dots,\Fsigned{B_{i-1}},\Fsigned{B_{i+1}},\dots,\Fsigned{B_m} &
\raisebox{-8pt}[0pt][0pt]{($p \in L, L \in \Loops{P}, \EB{L}{P}=\{B_1,\dots,B_m\}$)}
\\\cline{1-1}
\Tsigned{B_i}
\end{array}
}
\]
\item<2-> \structure{Example}
\[
\begin{array}{cl}
a \leftarrow a \\
a \leftarrow \neg b \\
\Tsigned{a}&
\raisebox{-8pt}[0pt][0pt]{($\EB{\{a\}}{P}=\{\{\neg b\}\}$)}
\\\cline{1-1}
\Tsigned{\{\neg b\}}
\end{array}
\]
\end{itemize}
\end{frame}
%----------------------------------------------------------------
\begin{frame}{Tableau rules for loops}
  \bigskip
  \begin{itemize}
  \item<1-> Tableau rules FL and BL ensure non-circular support for $\mathit{true}$ atoms
  \item<2-> For a loop $L$ such that $\EB{L}{P}=\{B_1,\dots,B_m\}$,\\
    they amount to implications of form
    \[
    \textstyle\bigvee_{p\in L} p \rightarrow B_1 \vee\dots\vee B_m
    \]
  \item<3-> Comparison to well-founded tableau rules yields
    \begin{itemize}\normalsize
    \item FL (plus FFA and FFB) is equivalent to WFN (plus FFB),
    \item BL cannot simulate inferences via WFJ
    \end{itemize}
  \end{itemize}
\end{frame}
%----------------------------------------------------------------
\begin{frame}{Relationship with loop formulas}
  \bigskip
  \begin{itemize}
  \item <1-> Tableau rules FL and BL essentially provide (straightforward) inferences from loop formulas
    \begin{itemize}
    \item Impractical to precompute exponentially many loop formulas
    \end{itemize}
    \medskip
  \item <2-> In practice, ASP solvers such as \textit{smodels} and \textit{clasp}
    \begin{itemize}
    \item exploit strongly connected components of positive atom\\ dependency graphs
      \begin{itemize}
      \item can be viewed as an interpolation of FL
      \end{itemize}
    \item <3-> do not directly implement BL (and neither WFJ)
      \begin{itemize}
      \item probably difficult to do efficiently
      \end{itemize}
    \item <4-> could simulate BL via FL/WFN by means of failed-literal detection (lookahead)
    \end{itemize}
  \end{itemize}
\end{frame}
% ----------------------------------------------------------------
\begin{frame}{Case analysis by $\textit{Cut}$}
  \bigskip
  \begin{itemize}
  \item <1-> Up to now, all tableau rules are deterministic
  \item [] That is, rules extend a single branch but cannot create sub-branches
    \smallskip
  \item <2-> In general, closing a branch leads to a partial assignment
    \smallskip
  \item <3-> Case analysis is done by $\textit{Cut}[{\mathcal{C}}]$ where
    ${\mathcal{C}} \subseteq \atom{P} \cup \body{P}$
  \end{itemize}
\end{frame}
% ----------------------------------------------------------------
\begin{frame}{Case analysis by $\textit{Cut}$}
\begin{itemize}
\item \structure{Prerequisites} None
\item \structure{Consequence} Two alternative (complementary) entries
\item \structure{Tableau Rule} \textnormal{$\textit{Cut}[\mathcal{C}]$}~\\[-5mm]
\[
\begin{array}{cl}
&
\raisebox{-8pt}[0pt][0pt]{($v \in {\mathcal{C}}$)}
\\\cline{1-1}
\Tsigned{v} \;\; \mid \;\; \Fsigned{v}
\end{array}
\]
\item<2-> \structure{Examples}
\[
\begin{array}{cl}
a \leftarrow \neg b \\
b \leftarrow \neg a &
\raisebox{-8pt}[0pt][0pt]{(${\mathcal{C}} = \atom{P}$)}
\\\cline{1-1}
\Tsigned{a} \;\; \mid \;\; \Fsigned{a}
\\[10pt]
a \leftarrow \neg b \\
b \leftarrow \neg a &
\raisebox{-8pt}[0pt][0pt]{(${\mathcal{C}} = \body{P}$)}
\\\cline{1-1}
\Tsigned{\{\neg b\}} \;\; \mid \;\; \Fsigned{\{\neg b\}}
\end{array}
\]
\end{itemize}
\end{frame}
%----------------------------------------------------------------
\begin{frame}{Well-known tableau calculi}
  \begin{itemize}
  \item <1-> Fitting's operator $\mathbf{\Phi}$ applies forward propagation without sophisticated
    unfounded set checks
    \[
    {\mathcal{T}}_{\mathbf{\Phi}} = \{\textit{FTB},\textit{FTA},\textit{FFB},\textit{FFA}\}
    \]
  \item <2-> Well-founded operator $\mathbf{\Omega}$ replaces negation of single atoms with negation
    of unfounded sets
    \[
    {\mathcal{T}}_{\mathbf{\Omega}} = \{\textit{FTB},\textit{FTA},\textit{FFB},\textit{WFN}\}
    \]
  \item <3-> ``Local'' propagation via a program's completion can be determined by elementary
    inferences on atoms and rule bodies
    \[
    {\mathcal{T}}_{\textit{completion}} =
    \{\textit{FTB},\textit{FTA},\textit{FFB},\textit{FFA},\textit{BTB},\textit{BTA},\textit{BFB},\textit{BFA}\}
    \]
  \end{itemize}
\end{frame}
% ----------------------------------------------------------------------
%
%%% Local Variables:
%%% mode: latex
%%% TeX-master: "../../main"
%%% End:
