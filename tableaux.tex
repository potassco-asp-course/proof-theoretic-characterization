% ----------------------------------------------------------------------
\begin{frame}{Motivation}
  \bigskip
  \begin{itemize}
  \item<1-> \structure{Goal} Analyze computations in ASP solvers
    \smallskip
  \item<1-> \structure{Wanted}
    A declarative and fine-grained instrument for \\ characterizing
    operations as well as strategies of ASP solvers
    \bigskip
  \item<2-> \structure{Idea}
    View stable model computations as derivations in\\ an inference system
  \item<3-> [] Consider \alert{Tableau-based proof systems} for analyzing ASP solving
\end{itemize}
\end{frame}
% ----------------------------------------------------------------------
\begin{frame}{Tableau calculi}
  \bigskip
  \begin{itemize}
  \item Traditionally, tableau calculi are used for
    \begin{itemize}
    \item automated theorem proving and
    \item proof theoretical analysis
    \end{itemize}
    in classical as well as non-classical logics
    \medskip
  \item \structure{General idea}
    Given an input, prove some property by decomposition
    Decomposition is done by applying deduction rules
  \item For details, see \emph{Handbook of Tableau Methods}, Kluwer, 1999
  \end{itemize}
\end{frame}
% ----------------------------------------------------------------------
\begin{frame}{General definitions}

\begin{itemize}
\item
A \alert{tableau} is a (mostly binary) tree
\item
A \alert{branch} in a tableau is a path from the root to a leaf
\item<2->
A branch containing $\gamma_1,\dots,\gamma_m$ can be extended by
applying\\ \alert{tableau rules} of form
\[
\begin{array}[t]{c}
\gamma_1,\dots,\gamma_m\\
\hline
\alpha_1\\[-1mm]
\vdots\\[-2mm]
\alpha_n
\end{array}
\qquad\qquad\pause[3]\qquad\qquad
\begin{array}[t]{c}
\gamma_1,\dots,\gamma_m\\
\hline
\beta_1\; \mid \; \dots \; \mid \;\beta_n
\end{array}
\qquad\qquad
\]
\item<2->
Rules of the former format append entries $\alpha_1,\dots,\alpha_n$ to the branch
\item<3->
Rules of the latter format create multiple sub-branches for $\beta_1,\dots,\beta_n$
\end{itemize}
\end{frame}
% ----------------------------------------------------------------------
\begin{frame}{Example}
  \begin{itemize}
  \item<1-> A simple tableau calculus for proving unsatisfiability of propositional
    formulas, composed from $\neg$, $\wedge$, and $\vee$, consists of rules
    \[
    \begin{array}[t]{c}
      \neg\neg \alpha\\
      \hline
      \alpha
    \end{array}
    \qquad
    \begin{array}[t]{c}
      \alpha_1 \wedge \alpha_2 \\
      \hline
      \alpha_1\\[-1mm]
      \alpha_2
    \end{array}
    \qquad
    \begin{array}[t]{c}
      \beta_1 \vee \beta_2 \\
      \hline
      \beta_1 \;\; \mid \;\; \beta_2
    \end{array}
    \]
  \item<2-> All rules are semantically valid, when interpreting entries in a branch conjunctively and distinct (sub-)branches as connected disjunctively
  \item<3->
    A propositional formula $\varphi$ % (composed from $\neg$, $\wedge$, and $\vee$)
    is unsatisfiable iff
    there is a tableau with\\ $\varphi$ as the root node such that
    \begin{enumerate}
    \item all other entries can be produced by tableau rules and
    \item every branch contains some formulas $\alpha$ and $\neg\alpha$
    \end{enumerate}
  \end{itemize}
\end{frame}
% ----------------------------------------------------------------------
\begin{frame}{Example}

\[\hspace{1cm}
\begin{array}{lcccccr}
\textbf{(1)}
& & &
a \wedge ((\neg b \wedge (\neg a \vee b)) \vee  \neg\neg\neg a)
& & &
[\varphi]
\\
\onslide<2->\textbf{(2)}
& & &
\onslide<2->\alert<6,9>{a}
& & &
\onslide<2->[\text{1}]
\\
\onslide<2->\textbf{(3)}
& & &
\onslide<2->(\neg b \wedge (\neg a \vee b)) \vee  \neg\neg\neg a
& & &
\onslide<2->[\text{1}]
\\[2mm]
& & &
\hspace{-10cm}
\begin{array}{lcccrcccclcccr}
\onslide<3->\textbf{(4)}
& &
\onslide<3->\neg b \wedge (\neg a \vee b)
& &
\onslide<3->[\text{3}]
& & & & &
\onslide<3->\textbf{(9)}
& &
\onslide<3->\neg\neg\neg a
& &
\onslide<3->[\text{3}]
\\
\onslide<4->\textbf{(5)}
& &
\onslide<4->\alert<7>{\neg b}
& &
\onslide<4->[\text{4}]
& & & & &
\onslide<8->\textbf{(10)}
& &
\onslide<8->\alert<9>{\neg a}
& &
\onslide<8->[\text{9}]
\\
\onslide<4->\textbf{(6)}
& &
\onslide<4->\neg a \vee b
& &
\onslide<4->[\text{4}]
\\[1mm]
& &
\hspace{-5cm}
\begin{array}{lcrcccclcr}
\onslide<5->\textbf{(7)}
&
\onslide<5->\alert<6>{\neg a}
&
\onslide<5->[\text{6}]
& & & &
\onslide<5->\textbf{(8)}
&
\onslide<5->\alert<7>{b}
&
\onslide<5->[\text{6}]
\end{array}
\hspace{-5cm}
\end{array}
\hspace{-10cm}
\end{array}
\]
\begin{itemize}
\item <10->
All three branches of the tableau are contradictory (cf 2, 5, 7, 8, 10)
\item <11-> Hence, $a \wedge ((\neg b \wedge (\neg a \vee b)) \vee  \neg\neg\neg a)$ is unsatisfiable
\end{itemize}
\end{frame}
%------------------------------------------------------------


%----------------------------------------------------------------
\begin{frame}{Tableau calculi characterizing ASP solvers}
  \begin{itemize}
  \item <1-> ASP solvers combine propagation with case analysis
  \item <2-> We obtain the following tableau calculi characterizing
    \begin{eqnarray*}
      {\mathcal{T}}_{\textit{cmodels-1}}
      & = &
      {\mathcal{T}}_{\textit{completion}} \cup \{\textit{Cut}[\atom{P} \cup \body{P}]\} \\
      {\mathcal{T}}_{\textit{assat}}
      & = &
      {\mathcal{T}}_{\textit{completion}} \cup \{\textit{FL}\} \cup \{\textit{Cut}[\atom{P} \cup \body{P}]\}  \\
      {\mathcal{T}}_{\textit{smodels}}
      & = &
      {\mathcal{T}}_{\textit{completion}} \cup \{\textit{WFN}\}\cup \{\textit{Cut}[\atom{P}]\} \\
      {\mathcal{T}}_{\textit{noMoRe}}
      & = &
      {\mathcal{T}}_{\textit{completion}} \cup \{\textit{WFN}\}\cup \{\textit{Cut}[\body{P}]\} \\
      {\mathcal{T}}_{\textit{nomore}^{++}}
      & = &
      {\mathcal{T}}_{\textit{completion}} \cup \{\textit{WFN}\}\cup \{\textit{Cut}[\atom{P} \cup \body{P}]\}
    \end{eqnarray*}
  \item<3-> SAT-based ASP solvers,
        \textit{assat} and \textit{cmodels},\\
        incrementally add loop formulas to a program's completion
  \item<4->  Native ASP solvers,
        \textit{smodels}, \textit{dlv}, \textit{noMoRe}, and \textit{nomore++},
        essentially differ only in their $\textit{Cut}$ rules
  \end{itemize}
\end{frame}
%------------------------------------------------------------


%%% Local Variables:
%%% mode: latex
%%% TeX-master: "../../main"
%%% End:
